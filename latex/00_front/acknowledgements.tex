\chapter*{Acknowledgements}

Doctoral work in sociology is, ironically, a fairly solitary affair.
However, if there’s one thing our discipline teaches us, is that our accomplishments are never only our own but the outcome of the social worlds we inhabit. Numerous people and institutions provided invaluable support, advice and feedback without which I would have never been able to complete this work.

First of all, I am grateful for the work of the historians who toil away in archives so we don’t have to. I’m specially indebted to the Old Bailey Online Project, and in particular to Sharon Howard for answering my questions and giving me direct access to the consolidated sources of the electronic edition of the Proceedings.
I’m also thankful for the work and help of the many hackers and programmers who develop the tools I used in completion of this work, particularly Christian Grün of the Basex project, Richard Eckart de Castilho of the DKPro project and Mike McCandles of the Lucene project.

I am also grateful to the institutions that provided me with financial or logistical support during all my years at Columbia: the Fulbright Commission, the Chilean National Science and Technology Comission, the Andrew W. Mellon Foundation and specially the Interdisciplinary Center for Innovative Theory and Empirics that was my academic domicile in the last few years.

My greatest debt is to Peter Bearman, who never lost faith in my work even when I had one.
His constant encouragement and enthusiasm about my ideas were the real force that allowed my inadequate efforts to complete this project.
In addition to his role as my advisor and his support as director of INCITE, I owe Peter for a way of thinking about the social that is exciting, fruitful, fun, rigorous and true.

I owe a great deal to the many colleagues with whom I have had the privilege of working, sharing ideas and arguing.
In particular, I am thankful to Alix Rule, Ryan Hagen and the rest of the participants in the Extreme Sociology workshop.
Many thanks to Seymour Spilerman for his guidance in the first few years of my doctoral training, and to Yinon Cohen and Thom DiPrete and the rest of the members of the Center for the Study of Wealth and Inequality for their warm welcome and stimulating and collegial conversation.
I am also thankful to Rozlyn Redd, Phillip Brandt, Josh Whitford, Shamus Kahn, David Stark and the participants in the Collaborative Organization and Digital Ecologies Seminar.
A special thanks goes to William McCallister, Audrey Augenbraum and the 2014-2016 cohorts of the Mellon Interdisciplinary Fellows Program. Another special thanks goes to Dora Arenas, Afton Battle and Kimesha Wilson to whom I owe my survival at Columbia.

A special thanks goes to those who I met as colleagues but became family: Olivia Nicol, Luciana de Souza Leão, Moran Levy, Adam Obeng and specially Sarah Sachs, from whom I learned more than she can imagine.

Finally, I am for ever in debt to my parents for their constant love and support, my siblings for showing me the way ahead, and to Tatiana Reyes por ser el rayo que ilumina el camino.
