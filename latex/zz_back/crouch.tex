\chapter{The Trial of Arthur Crouch}
\label{app:crouch}

This is the text corresponding to trial record \code{t18940305-268}, the trial of Arthur Crouch for feloniously wounding Mary Crouch with intent to do grievious bodily harm.
This rendering of the text corresponds to the character stream after acquisition by the SAX parser, including all amendments and corrections to character offset errors induced from the mixed-content information contained in the original XML file.
Side notes indicate the provenance metadata originally encoded as in-line markup in the XML source.
These are passed to the index as part of a separate field, retaining its offset location into the source character stream, such that reference information is retained, but this is finally used only for purposes of document sampling at the paragraph level.

This particular case includes an offence description that corresponds to a violent subcategory (``wounding''), has no female defendant, has a female victim, has no groups, and has a non-corporal punishment.
In consequence, the provenance metadata for trial account \code{t18940305-268} include its contained paragraphs in $C_{violence}$, $C_{female}$, $\sim C_{groups}$ and $\sim C_{corporal}$.

All paragraphs containing an ``Offence'', ``Verdict'' or ``Punishment'' annotation are included in $C_{legal}$ and by definition included in $\sim C_{testimony}$.
All other paragraphs belong to $C_{trial}$ and are thus included in $C_{testimony}$.

\begin{center}
    ***
\end{center}

\newgeometry{left=6cm}

{
    \setlength{\parindent}{0em}
    \setlength{\parskip}{1em}
    \renewcommand*{\sideparfont}{\footnotesize\itshape}

    \sidepar{
        Pers. name:\\defendant;\\male; 37
        \\
        Offence:\\breaking peace;\\wounding
    }
    268. \textsc{Arthur Crouch} (37), was indicted for feloniously wounding Mary Crouch, with intent to do grievous bodily harm.

    \textsc{Mr. Grazebrook} \textit{Prosecuted} and \textsc{Mr. Keith Frith} \textit{Defended}.

    \sidepar{
        Pers. name:\\victim; female
    }
    \textsc{Mary Crouch}. I am the prisoner's wife, and live at 88, Cirencester Street, Harrow Road
    ---on the night my husband assaulted me I had been drinking, but I was a bit sober when he came home
    ---I can't exactly remember what night it was; I think it was in the middle of last month, on a Thursday, I think
    ---his tea was not ready, and he said I had been drinking
    ---I said I had not
    ---he said, ``Don't irritate me''
    ---I flew at him in a temper, and he deliberately took up my legs and threw me down on the floor, on the landing of the top floor
    ---our room is the top back room of the fourth floor
    ---I said, ``Wait till I get up''
    ---I got up; he went out, as I thought into the next room to get a hammer or Something, as when we are in a temper and have had drink we don't know what we do; I should hit him with the hammer as he would hit me
    ---I was on the landing; the window is rather low, and as he pushed me and I went out at the back landing window
    ---that window is very low from the floor on the inside
    ---I am not sure that I said before the Magistrate, ``He lifted me up and threw me through the landing window. I was in a temper at the time, he pushed me through the window''
    ---I think that is a mistake
    ---this is my deposition
    ---I signed it
    ---I did say that before the Magistate
    ---I was never in such a place before
    ---I don't know how much drink I had had on this day
    ---my husband had never threatened me before this
    ---when in drink he might say a word, but not when sober
    ---after I got through the window I fell on to the cistern, the lid gave way, and I got into it, and I got out myself
    ---my landlady, Mrs. Porter, and a man came across the road
    ---Mrs. Porter did not see me in the cistern, I was leaning on a plank, and she opened the landing window and helped me to get out, and got a bit of rag and tied round my finger
    ---I was not much hurt, only my finger was bleeding
    ---the surgeon dressed it at the station
    ---I told the Magistrate that the prisoner had frequently threatened to swing for me, and to throw me out of the window.

    \textit{Cross-examined}. I have said that this was an accident
    ---I do not suggest that he deliberately threw me out of the window; the cut on my finger was done by the glass in trying to save myself
    ---he was under the influence of drink or he would not have done it; I might have slipped
    ---we' have been married eighteen years
    ---when in drink we are both rather violent
    ---I thought he went to fetch the hammer; if he had hit me with it I should have hit him back
    ---I have said that it was a pure accident, through my having drink; but I leave it to you, gentlemen
    ---in my temper I told the policeman to take him
    ---we always lived happy and comfortable when he was a teetotaler, there could not be a better husband; I don't want to go on with the case.

    \sidepar{
        Pers. name:\\witness;\\male
    }
    \textsc{William Hawse}. I am a cellarman, and live at 14, Senior Street, directly opposite the prisoner
    ---on the evening of February 15th I heard a very loud smash, as if of glass
    ---I went out on my leads and saw that the top landing window was smashed completely through, and the woman in the cistern
    ---I went across and helped her out
    ---Mrs. Porter went up with me
    ---she was lying in a helpless state, half in and half out, and bleeding
    ---she said, ``My husband has thrown me through the window''
    ---I did not see the prisoner
    ---the cistern is about twelve feet from the window
    ---the window was completely smashed out
    ---I saw nothing about the woman to warrant me in saying she was intoxicated
    ---she was very excited
    ---I tied up her hand
    ---the distance from the cistern to the ground was about twenty to twenty-two feet
    ---at the bottom the ground is concrete.

    \textit{Cross-examined}. She was quite up to her knees in water
    ---she is quite a stranger to me, only as a neighbour
    ---I had never spoken to her.

    By the \textsc{Court}. The window had a double sash, with two panes in each
    ---both panes were smashed, and the woodwork as well.

    \sidepar{
        Pers. name:\\witness;\\male
    }
    \textsc{William Spencer} (\textit{Policeman}). On the evening of the 15th February I was called to 88, Cirencester Street
    ---I found Hawes holding the prisoner by the hand
    ---he was binding it up
    ---it was bleeding very much
    ---from what I was told I went and saw the prisoner
    ---he was upstairs, about seven steps from the window
    ---the prosecutrix requested me to take the prisoner into custody for wilfully throwing her through the window
    ---he replied, ``Yes, and I wish she had stopped there, the b
    ---'' he was sober
    ---I should say the prosecutrix appeared to have been drinking, but at the station she seemed to have recovered from it
    ---the window inside the landing is about two feet from the floor, and about twelve feet from the cistern.

    \textit{Cross-examined}. I was there with the inspector when he examined it
    ---the woodwork was broken away
    ---the cover of the cistern was broken in
    ---the prosecutrix is a big, heavy woman
    ---all the houses in the street are not kept in good repair
    ---there are rotten cistern covers
    ---I have known the prosecutrix nearly twenty years
    ---I have never seen her drunk or the worse for drink.

    \sidepar{
        Pers. name:\\witness;\\male
    }
    \textsc{James Bristow} (\textit{Inspector X}). On the evening of 15th February, I was in charge of the station when the prisoner was brought there
    ---I told him he would be charged with violently assaulting his wife by throwing her through the landing window, whereby she fell through into the cistern, causing her grievous bodily harm
    ---he made no answer; all he said was, ``Can I have bail?''
    ---I afterwards went and examined the house with a constable
    ---I found that the stairs where the prosecutrix stood were about two feet from the window
    ---the window itself is about three and a half feet wide
    ---the woodwork was broken
    ---the distance from the cistern to the ground is about twenty-two feet
    ---if that wood had not given way, she must have rolled off to the ground
    ---I saw her at the station
    ---she was excited, but certainly not drunk
    ---the prisoner was perfectly sober.

    \textit{Cross-examined}. She was very much excited and seemed to have been drinking
    ---I particularly examined the window sashes
    ---they were very old, and in very bad repair
    ---I don't think they would give way readily
    ---the stairs are very rotten, and the house generally in bad repair
    ---I tried the next window on the landing, and pushed it very hard and could not move it.

    \sidepar{
        Pers. name:\\witness;\\male
    }
    \textsc{George Robertson} (\textit{M. D., and Divisional Surgeon of Police at Kilburn Park Road}). On the 15th of last month, at 9.15 p.m., I saw the prosecutrix at the Harrow Road Police-station
    ---she was suffering from a severe lacerated wound of the left hand, about a square inch of skin and flesh being torn out
    ---it was not a dangerous wound, but it was a severe wound
    ---it appeared to be such an injury as would be caused by glass
    ---it was dirty
    ---she was very much excited, but appeared to be perfectly sober
    ---she may have been drinking
    ---she certainly knew what she was about.

    \textit{Cross-examined}. She had no signs of collapse.

    \sidepar{
        Pers. name:\\witness;\\female
    }
    \textsc{Rosina Porter} (\textit{Examined by} \textsc{Mr. Keith Frith}). The prisoner and his wife lived with me six months
    ---they appeared to lead a happy life
    ---I have not seen the prisoner above four times since they lived there
    ---he appeared to be a quiet, hard-working man
    ---they lived at the top of the house; I was down in the kitchen
    ---I never heard them quarrel
    ---I have seen her under the influence of drink
    ---on this day she had been out all day drinking; she was rowing with a neighbour next door all day
    ---he was locked up, and when he came home at night there was no tea or fire; there was a little bit of steak there, not cooked
    ---the window out of which the woman either was thrown or fell was only two or three feet from the landing, and the landing was about three feet four inches wide
    ---in a scuffle a man or woman could easily fall through if slightly pushed
    ---Mrs. Crouch was tight when she was quarrelling with the woman next door, but she had got better when her husband came home
    ---I was bound over as a witness for the prosecution
    ---the prisoner is a very good principled young man
    ---he gets tight too, but she more frequently, and she is a very bad-tempered woman
    ---the landing is so narrow that in a scuffle the woman might have gone through by accident without a push
    ---I did not see her in the cistern; she was leaning on a plank that goes across
    ---I held the light while her finger was bound up
    ---she was not drunk enough to tumble through the window herself
    ---she was not drunk then.

    The prisoner's statement before the Magistrate: ``We are both hardworking people. We both drink hard together. It is all through the drink that has caused this trouble. I hope it will be looked over; it is the first time I have gone to prison.''

    \sidepar{
        Verdict:\\guilty;\\lesser offence
    }
    \textsc{Guilty} of unlawful wounding.

    \sidepar{
        Punishment:\\imprison;\\hard labour
    }
    ---Six Months' Hard Labour.
}
\restoregeometry
